\documentclass[11pt,letterpaper]{article}

% Load some basic packages that are useful to have
% and that should be part of any LaTeX installation.
%
% be able to include figures
\usepackage{graphicx}
% get nice colors
\usepackage{xcolor}

% change default font to Palatino (looks nicer!)
\usepackage[latin1]{inputenc}
\usepackage{mathpazo}
\usepackage[T1]{fontenc}
% load some useful math symbols/fonts
\usepackage{latexsym,amsfonts,amsmath,amssymb}

% comfort package to easily set margins
\usepackage[top=1in, bottom=1in, left=1in, right=1in]{geometry}

% control some spacings
%
% spacing after a paragraph
\setlength{\parskip}{.15cm}
% indentation at the top of a new paragraph
\setlength{\parindent}{0.0cm}


\begin{document}

\begin{center}
\Large
Ay190 -- Worksheet 3\\
Anthony Alvarez\\
Date: \today
\end{center}

\section{Root Finding: Eccentricity Anomality}
\subsection{Low Ecentricity}
First in the not too eccentric orbit we find that using the secant method, 
where the derivaitve is estimated using a secant, that it only takes 4 steps
to get to a relative error of less than $10^{-10}$. We also find that 

\begin{center}
\begin{tabular}{| c | c | c | c |}
  \hline
  t & x & y & counter \\
  \hline                       
  91 & -16898.4 & 1495695.9 & 4\\
  182 & -1495915.5 &  15897.6& 4\\
  273 &  -49209.3& -1494981.9 &4 \\
  \hline  
\end{tabular}
\end{center}

\subsection{High Ecentricity}

When we increase the ecentricity of the orbit we find that it does take slightly
longer to converge but it does not take dramatically longer. 
\begin{center}
\begin{tabular}{| c | c | c | c |}
  \hline
  t & x & y & counter \\
  \hline                       
  91 & -1004146.3& 49.5 & 7\\
  182 & -1495978.1 &  0.36 & 7\\
  273 & -1018362.04 & -49.0 &7 \\
  \hline  
\end{tabular}
\end{center}

We could accelerate convergance by TKTK
\end{document}
